% Options for packages loaded elsewhere
\PassOptionsToPackage{unicode}{hyperref}
\PassOptionsToPackage{hyphens}{url}
%
\documentclass[
]{article}
\usepackage{lmodern}
\usepackage{amsmath}
\usepackage{ifxetex,ifluatex}
\ifnum 0\ifxetex 1\fi\ifluatex 1\fi=0 % if pdftex
  \usepackage[T1]{fontenc}
  \usepackage[utf8]{inputenc}
  \usepackage{textcomp} % provide euro and other symbols
  \usepackage{amssymb}
\else % if luatex or xetex
  \usepackage{unicode-math}
  \defaultfontfeatures{Scale=MatchLowercase}
  \defaultfontfeatures[\rmfamily]{Ligatures=TeX,Scale=1}
\fi
% Use upquote if available, for straight quotes in verbatim environments
\IfFileExists{upquote.sty}{\usepackage{upquote}}{}
\IfFileExists{microtype.sty}{% use microtype if available
  \usepackage[]{microtype}
  \UseMicrotypeSet[protrusion]{basicmath} % disable protrusion for tt fonts
}{}
\makeatletter
\@ifundefined{KOMAClassName}{% if non-KOMA class
  \IfFileExists{parskip.sty}{%
    \usepackage{parskip}
  }{% else
    \setlength{\parindent}{0pt}
    \setlength{\parskip}{6pt plus 2pt minus 1pt}}
}{% if KOMA class
  \KOMAoptions{parskip=half}}
\makeatother
\usepackage{xcolor}
\IfFileExists{xurl.sty}{\usepackage{xurl}}{} % add URL line breaks if available
\IfFileExists{bookmark.sty}{\usepackage{bookmark}}{\usepackage{hyperref}}
\hypersetup{
  pdftitle={Correlation: Variance, Covariance, and Plotting},
  pdfauthor={Kim Wiggins},
  hidelinks,
  pdfcreator={LaTeX via pandoc}}
\urlstyle{same} % disable monospaced font for URLs
\usepackage[margin=1in]{geometry}
\usepackage{color}
\usepackage{fancyvrb}
\newcommand{\VerbBar}{|}
\newcommand{\VERB}{\Verb[commandchars=\\\{\}]}
\DefineVerbatimEnvironment{Highlighting}{Verbatim}{commandchars=\\\{\}}
% Add ',fontsize=\small' for more characters per line
\usepackage{framed}
\definecolor{shadecolor}{RGB}{248,248,248}
\newenvironment{Shaded}{\begin{snugshade}}{\end{snugshade}}
\newcommand{\AlertTok}[1]{\textcolor[rgb]{0.94,0.16,0.16}{#1}}
\newcommand{\AnnotationTok}[1]{\textcolor[rgb]{0.56,0.35,0.01}{\textbf{\textit{#1}}}}
\newcommand{\AttributeTok}[1]{\textcolor[rgb]{0.77,0.63,0.00}{#1}}
\newcommand{\BaseNTok}[1]{\textcolor[rgb]{0.00,0.00,0.81}{#1}}
\newcommand{\BuiltInTok}[1]{#1}
\newcommand{\CharTok}[1]{\textcolor[rgb]{0.31,0.60,0.02}{#1}}
\newcommand{\CommentTok}[1]{\textcolor[rgb]{0.56,0.35,0.01}{\textit{#1}}}
\newcommand{\CommentVarTok}[1]{\textcolor[rgb]{0.56,0.35,0.01}{\textbf{\textit{#1}}}}
\newcommand{\ConstantTok}[1]{\textcolor[rgb]{0.00,0.00,0.00}{#1}}
\newcommand{\ControlFlowTok}[1]{\textcolor[rgb]{0.13,0.29,0.53}{\textbf{#1}}}
\newcommand{\DataTypeTok}[1]{\textcolor[rgb]{0.13,0.29,0.53}{#1}}
\newcommand{\DecValTok}[1]{\textcolor[rgb]{0.00,0.00,0.81}{#1}}
\newcommand{\DocumentationTok}[1]{\textcolor[rgb]{0.56,0.35,0.01}{\textbf{\textit{#1}}}}
\newcommand{\ErrorTok}[1]{\textcolor[rgb]{0.64,0.00,0.00}{\textbf{#1}}}
\newcommand{\ExtensionTok}[1]{#1}
\newcommand{\FloatTok}[1]{\textcolor[rgb]{0.00,0.00,0.81}{#1}}
\newcommand{\FunctionTok}[1]{\textcolor[rgb]{0.00,0.00,0.00}{#1}}
\newcommand{\ImportTok}[1]{#1}
\newcommand{\InformationTok}[1]{\textcolor[rgb]{0.56,0.35,0.01}{\textbf{\textit{#1}}}}
\newcommand{\KeywordTok}[1]{\textcolor[rgb]{0.13,0.29,0.53}{\textbf{#1}}}
\newcommand{\NormalTok}[1]{#1}
\newcommand{\OperatorTok}[1]{\textcolor[rgb]{0.81,0.36,0.00}{\textbf{#1}}}
\newcommand{\OtherTok}[1]{\textcolor[rgb]{0.56,0.35,0.01}{#1}}
\newcommand{\PreprocessorTok}[1]{\textcolor[rgb]{0.56,0.35,0.01}{\textit{#1}}}
\newcommand{\RegionMarkerTok}[1]{#1}
\newcommand{\SpecialCharTok}[1]{\textcolor[rgb]{0.00,0.00,0.00}{#1}}
\newcommand{\SpecialStringTok}[1]{\textcolor[rgb]{0.31,0.60,0.02}{#1}}
\newcommand{\StringTok}[1]{\textcolor[rgb]{0.31,0.60,0.02}{#1}}
\newcommand{\VariableTok}[1]{\textcolor[rgb]{0.00,0.00,0.00}{#1}}
\newcommand{\VerbatimStringTok}[1]{\textcolor[rgb]{0.31,0.60,0.02}{#1}}
\newcommand{\WarningTok}[1]{\textcolor[rgb]{0.56,0.35,0.01}{\textbf{\textit{#1}}}}
\usepackage{graphicx}
\makeatletter
\def\maxwidth{\ifdim\Gin@nat@width>\linewidth\linewidth\else\Gin@nat@width\fi}
\def\maxheight{\ifdim\Gin@nat@height>\textheight\textheight\else\Gin@nat@height\fi}
\makeatother
% Scale images if necessary, so that they will not overflow the page
% margins by default, and it is still possible to overwrite the defaults
% using explicit options in \includegraphics[width, height, ...]{}
\setkeys{Gin}{width=\maxwidth,height=\maxheight,keepaspectratio}
% Set default figure placement to htbp
\makeatletter
\def\fps@figure{htbp}
\makeatother
\setlength{\emergencystretch}{3em} % prevent overfull lines
\providecommand{\tightlist}{%
  \setlength{\itemsep}{0pt}\setlength{\parskip}{0pt}}
\setcounter{secnumdepth}{-\maxdimen} % remove section numbering
\ifluatex
  \usepackage{selnolig}  % disable illegal ligatures
\fi

\title{Correlation: Variance, Covariance, and Plotting}
\author{Kim Wiggins}
\date{}

\begin{document}
\maketitle

Add libraries and load the Rdata file from the session.

Park names, park sizes and visitors printed below:

\begin{Shaded}
\begin{Highlighting}[]
\NormalTok{parks}
\end{Highlighting}
\end{Shaded}

\begin{verbatim}
##                   size visitors
## Arcadia           47.4     2.05
## Bryce Canyon      35.8     1.02
## Cuyahoga Valley   32.9     2.53
## Everglades      1508.5     1.23
## Grand Canyon    1217.4     4.40
## Grand Tenton     310.0     2.46
## Great Smoky      521.8     9.19
## Hot Springs        5.6     1.34
## Olympic          922.7     3.14
## Mount Rainier    235.6     1.17
## Rocky Mountain   265.8     2.80
## Shenandoah       199.0     1.09
## Yellostone      2219.8     2.84
## Yosemite         761.3     3.30
## Zion             146.6     2.59
\end{verbatim}

Use the \texttt{cov()} and \texttt{cor()} commands to view the sample
covariance matrix and the sample correlation matrices.

\emph{Covariance matrix, Parks data}

\begin{Shaded}
\begin{Highlighting}[]
\FunctionTok{cov}\NormalTok{(parks)}
\end{Highlighting}
\end{Shaded}

\begin{verbatim}
##                 size   visitors
## size     424177.2998 228.416524
## visitors    228.4165   4.132295
\end{verbatim}

\emph{Correlation Matrix, Parks Data}

\begin{Shaded}
\begin{Highlighting}[]
\FunctionTok{cor}\NormalTok{(parks)}
\end{Highlighting}
\end{Shaded}

\begin{verbatim}
##               size  visitors
## size     1.0000000 0.1725274
## visitors 0.1725274 1.0000000
\end{verbatim}

The correlation matrix displays the Pearson correlation coefficient to
demonstrate variable relationship strength. As the scaled form of
covariance, correlation coefficients are standardized (non-scalar) and
dimensionless (unit-free), and can thus be interpreted easily on a scale
from -1 to +1. Covariance values vary from \(-\infinity\) to to positive
infinity. While covariance indicates the direction of the relationship
only, correlation indicates direction and strength, resulting in a
proportion metric measuring how much on average these variables differ
with regard to one another.

Because the covariance matrix isn't scaled, we can only interpret
direction. The size of the park is measured on a different scale than
the visitor count, so interpretation regarding strength of the direction
is not appropriate using the covariance matrix. In this case, the
covariance matrix tells us that the relationship is positive, but the
correlation coefficient of 0.173 being so close to 0 indicates that the
linear relationship is not strong. The variables may have some other
strong relationship that is not linear, and we must take care to not
interpret a 0 (or values close to 0) as indicating lack of any
relationship - only lack of a strong linear one.

We can also draw a matrix of plots containing histograms and boxplots of
park sizes and number of visitors, making sure to give appropriate
titles and axes labels for the histograms and clearly interpret the
graph/s.

\begin{Shaded}
\begin{Highlighting}[]
\FunctionTok{layout}\NormalTok{(}\FunctionTok{matrix}\NormalTok{(}\DecValTok{1}\SpecialCharTok{:}\DecValTok{2}\NormalTok{, }\AttributeTok{nc=}\DecValTok{2}\NormalTok{))}
\CommentTok{\# 1 by 2 rows}
\FunctionTok{boxplot}\NormalTok{(parks}\SpecialCharTok{$}\NormalTok{size, }\AttributeTok{main =} \StringTok{"Size"}\NormalTok{)}
\FunctionTok{boxplot}\NormalTok{(parks}\SpecialCharTok{$}\NormalTok{visitors, }\AttributeTok{main =} \StringTok{"Visitors"}\NormalTok{)}
\end{Highlighting}
\end{Shaded}

\includegraphics{Corr_files/figure-latex/unnamed-chunk-4-1.pdf}
Examination of the boxplots indicate two outliers: one for each
variable. Visual inspection of the elements reveal the potential
outliers to be Yellowstone for Size, and Great Smoky for visitors.\\
Be cautious to consider these \emph{potential} outliers - they could be
anomalies or belong to a bimodal distribution and actually be quite
typical of the data's behavior.

Histograms

\begin{Shaded}
\begin{Highlighting}[]
\FunctionTok{par}\NormalTok{(}\AttributeTok{mfrow=}\FunctionTok{c}\NormalTok{(}\DecValTok{2}\NormalTok{,}\DecValTok{1}\NormalTok{))}
\FunctionTok{hist}\NormalTok{(parks}\SpecialCharTok{$}\NormalTok{size, }\AttributeTok{main =} \StringTok{"Park Acreage, in Thousands"}\NormalTok{, }\AttributeTok{xlab =} \StringTok{"Size"}\NormalTok{, }\AttributeTok{prob=}\NormalTok{ T)}
\FunctionTok{hist}\NormalTok{(parks}\SpecialCharTok{$}\NormalTok{visitors, }\AttributeTok{main =} \StringTok{"Park Annual Visitors, in Thousands"}\NormalTok{, }\AttributeTok{xlab =} \StringTok{"Visitors"}\NormalTok{, }\AttributeTok{prob =}\NormalTok{ T)}
\end{Highlighting}
\end{Shaded}

\includegraphics{Corr_files/figure-latex/unnamed-chunk-5-1.pdf}

Histograms indicate that most values for Size fall within the 0-1000
range. Visitor ranges indicate the majority of observations fall between
1 and 4, with a potential outlier in the 8-10 range.

Explore for possible outliers. Use appropriate graphs to draw
conclusions. Calculate correlation coefficient between park size and
number of park visitors before and after removing possible outliers.
Comment on your findings.

\begin{Shaded}
\begin{Highlighting}[]
\NormalTok{outpark }\OtherTok{=} \FunctionTok{match}\NormalTok{(parkname }\OtherTok{\textless{}{-}} \FunctionTok{c}\NormalTok{(}\StringTok{"Great Smoky"}\NormalTok{, }\StringTok{"Yellostone"}\NormalTok{), }\FunctionTok{rownames}\NormalTok{(parks))}
\FunctionTok{bvbox}\NormalTok{(parks, }\AttributeTok{mtitle =} \StringTok{"Bivariate Boxplot, Parks"}\NormalTok{, }\AttributeTok{xlab =}\NormalTok{ size.lab, }\AttributeTok{ylab =}\NormalTok{ visitors.lab)}
\FunctionTok{text}\NormalTok{(parks}\SpecialCharTok{$}\NormalTok{size[outpark], parks}\SpecialCharTok{$}\NormalTok{visitors[outpark], }\AttributeTok{labels =}\NormalTok{ parkname, }\AttributeTok{pos =} \FunctionTok{c}\NormalTok{(}\DecValTok{2}\NormalTok{,}\DecValTok{2}\NormalTok{,}\DecValTok{4}\NormalTok{,}\DecValTok{2}\NormalTok{,}\DecValTok{2}\NormalTok{))}
\end{Highlighting}
\end{Shaded}

\includegraphics{Corr_files/figure-latex/unnamed-chunk-6-1.pdf}

The outliers are labeled and will be removed from future analysis. In
future cases, that is likely not the best practice, but for the purposes
of comparing before and after correlations here, we will remove them.

After exluding the outliers, the correlation coefficient increases from
0.173 to 0.399.

\begin{Shaded}
\begin{Highlighting}[]
\FunctionTok{with}\NormalTok{(parks, }\FunctionTok{cor}\NormalTok{(size, visitors))}
\end{Highlighting}
\end{Shaded}

\begin{verbatim}
## [1] 0.1725274
\end{verbatim}

\begin{Shaded}
\begin{Highlighting}[]
\FunctionTok{with}\NormalTok{(parks, }\FunctionTok{cor}\NormalTok{(size[}\SpecialCharTok{{-}}\NormalTok{outpark], visitors[}\SpecialCharTok{{-}}\NormalTok{outpark]))}
\end{Highlighting}
\end{Shaded}

\begin{verbatim}
## [1] 0.398539
\end{verbatim}

\end{document}
